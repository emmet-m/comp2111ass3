\documentclass[a4paper]{scrartcl}
\usepackage[l2tabu,orthodox]{nag}% Old habits die hard. All the same, there are commands, classes and packages which are outdated and superseded. nag provides routines to warn the user about the use of those.

\usepackage[all,error]{onlyamsmath}% Error on deprecated math commands like $$ $$.
\usepackage{fixltx2e}
\usepackage[strict=true]{csquotes}
\usepackage[usenames, dvipsnames]{color}
\usepackage[colorlinks=true]{hyperref}
\usepackage{2111defs2,2111theorems}
\usepackage{listings}

% Algortihms package for pretty code formatting!
\usepackage{algorithm2e}
\usepackage{algpseudocode}

\allowdisplaybreaks 

\title{\texttt{$\mathbb{C}$ase $\mathbb{S}$tudy: $\mathbb{H}$elp us oh $\mathcal{HELP}$}}
\author{$\mathbb{E}$mmet $\mathbb{M}$urray z5059840, $\mathbb{D}$anni $\mathbb{O}$vens z5059491}

% Danni's awesome and super helpful math things y@y
\newcommand{\N}{\mathbb{N}}
\newcommand{\Z}{\mathbb{Z}}
\newcommand{\C}{\mathcal{C}}
\newcommand{\Q}{\mathcal{Q}}
\newcommand{\refinedby}{\sqsubseteq} % THIS IS MY PERSONAL FAV ~refined by~
\newcommand\textlcsc[1]{\textsc{\MakeLowercase{#1}}} % pretty small uppercase letters
%This is actually the best thing we've probably ever done
\newcommand{\rc}[1]{ $\refinedby$ \quad \textbf{\textcolor{ForestGreen}{$\langle$ #1 $\rangle$}}}
\newcommand{\sset}[1]{ \llcorner #1 \lrcorner }

%I cna't live without this:
\newcommand{\squigglyBoy}{~}
\newcommand{\leftcurlyBoy}{\{}
\newcommand{\rightcurlyBoy}{\}}

\lstset{ %
  backgroundcolor=\color{white},   % choose the background color; you must add \usepackage{color} or \usepackage{xcolor}; should come as last argument
  basicstyle=\footnotesize,        % the size of the fonts that are used for the code
  breakatwhitespace=false,         % sets if automatic breaks should only happen at whitespace
  breaklines=true,                 % sets automatic line breaking
  captionpos=b,                    % sets the caption-position to bottom
  commentstyle=\color{DarkOrchid},    % comment style
  deletekeywords={...},            % if you want to delete keywords from the given language
  extendedchars=true,              % lets you use non-ASCII characters; for 8-bits encodings only, does not work with UTF-8
  frame=single,	                   % adds a frame around the code
  keepspaces=true,                 % keeps spaces in text, useful for keeping indentation of code (possibly needs columns=flexible)
  keywordstyle=\color{blue},       % keyword style
  language=C,                 % the language of the code
  morekeywords={*,...},            % if you want to add more keywords to the set
  numbers=left,                    % where to put the line-numbers; possible values are (none, left, right)
  numbersep=5pt,                   % how far the line-numbers are from the code
  numberstyle=\color{red}, % the style that is used for the line-numbers
  rulecolor=\color{black},         % if not set, the frame-color may be changed on line-breaks within not-black text (e.g. comments (green here))
  showspaces=false,                % show spaces everywhere adding particular underscores; it overrides 'showstringspaces'
  showstringspaces=false,          % underline spaces within strings only
  showtabs=false,                  % show tabs within strings adding particular underscores
  stepnumber=1,                    % the step between two line-numbers. If it's 1, each line will be numbered
  stringstyle=\color{red},     % string literal style
  tabsize=4,	                   % sets default tabsize to 2 spaces
  title=\lstname                   % show the filename of files included with \lstinputlisting; also try caption instead of title
}

% This is just fun:
\newcommand{\que}[1]{\langle #1 \rangle}
\newcommand{\explain}[1]{\textcolor{RoyalBlue}{\textit{#1}}}
\newcommand{\ab}[1]{\textcolor{WildStrawberry}{#1}}
\newcommand{\con}[1]{\textcolor{RoyalBlue}{#1}}
\newcommand{\tabb}{\null \quad}
\newcommand{\tabbb}{\null \quad \quad}
\newcommand{\tabbbb}{\null \quad \quad \quad}

% Document starts here
\begin{document}
\maketitle
\section{BFS Implementation}
%
\subsection{Defining an abstract queue}
We will begin by defining some abstract queue operations in our toy language: \\
\begin{center}
{\LARGE{\textsc{Queue:}}\normalsize}
\end{center}
\begin{equation*}
\mathcal{Q} :: ( N:\N, ~s:V_{t}^{*} ) \\
\end{equation*} \\
%
Where $N$ is the max size of the queue, $n$ is the current size, and $s$ is a sequence of queue values. \\
%
We introduce the notation of $\llcorner s \lrcorner$, meaning to treat the sequence $s$ as a set such that all the values of s are in the $\llcorner s \lrcorner$. For convenience, we define $|s|$ as equivalent to $|\llcorner s \lrcorner|$.\\
%
We can now define our 5 core abstract queue operations: \\ \\
%
\textbf{Initq:} \\
% Do we even need this? In Liam's example he just sets the queue to the empty queue.
$q : [True, ~ q = ( N, s) \land |s| = 0] \refinedby initq(q)$ \\
\textbf{Enq:} \\
$q : [|s| < N, ~ q = ( N, s) \land s = xs_0 \land |s| \neq 0] \refinedby enq(q, x)$ \\
\textbf{Deq:} \\
$q : [|sx| > 0 \land q = (N, sx), q = (N, s) \land s = s_0 ] \refinedby  deq(q)$ \\
\textbf{WhosNext:} \\
$x : [q = (N, sy), x = y] \refinedby x := whosNext(q)$ \\
\textbf{isEmpty:} \\
$b : [q = (N, s), b \iff |s| = 0 ] \refinedby b := isEmpty(q)$\\
%
\subsection{Refinement}
We begin by refining the provided specification of \textsc{Search}: \\ \\
%
\textbf{proc }{\textsc{Search}(\textbf{value} $t$, \textbf{value} $N$, \textbf{value} $k$, \textbf{result} $v$, \textbf{result} $f$) $\cdot$\\
\[ t, N, k, v, f: \left[
    \begin{array}{l} 
	\textsc{Tree}(t) \land \text{max}_{i\in\N} | \Gamma^i_t (r_t)  \cup \Gamma^{i+1}_t (r_t) | \leq N ,\\
	(f \land \exists w \in V_{t_0} (\kappa_{t_0}(w) = k_0 \land \lambda_{t_0} (w) = v)) ~ \lor \\
	(\neg f  \land \forall w \in V_{t_0} (\kappa_{t_0}(w) \neq k_0))
    \end{array} 
\right] \] \\
%
%
\rc{c-frame} \\ 
$\null \quad \nt{v, f: \left[
    \begin{array}{l} 
	\textsc{Tree}(t) \land \text{max}_{i\in\N} | \Gamma^i_t (r_t)  \cup \Gamma^{i+1}_t (r_t) | \leq N ,\\
	(f \land \exists w \in V_{t} (\kappa_{t}(w) = k \land \lambda_{t} (w) = v)) ~ \lor \\
	(\neg f  \land \forall w \in V_{t} (\kappa_{t}(w) \neq k))
    \end{array} 
\right]}{(1)}$ \\ \\
%
\\ In order to use a breadth first search along the tree, we need a queue. Thus we must create a queue variable and initialise it to an empty queue: \\  \\
%I LOC
(1) \rc{i-loc} \explain{q doesn't occur yet} \\
\null \quad \textbf{var } q : $\mathcal{Q}~\cdot$ 
$v, f, q \left[ 
~pre(1),~
post(1)~
\right]$ \\ \\
%
%
% SEQ
\rc{seq} \explain{so we can refine to initq}\\
$ \null \quad v, f, q : \left[ 
	~ pre(1), ~
	pre(1) \land q = (N,s) \land |s| = 0
\right]$; \\
$ \null \quad v, f, q : \left[ 	
	pre(1) \land q = (N,s) \land |s| = 0, ~
	post(1)
\right]$ \\ \\
%
%
\rc{initq} \explain{}\\
$ \null \quad initq(q);$ \\
$ \null \quad \nt{v, f, q : \left[ 	
	pre(1) \land q = (N,s) \land |s| = 0, ~
	post(1)
\right]}{(2)}$ \\ \\
%
%
Similarly, we push our initial element $r_t$ onto the queue: \\
%
(2) \rc{seq, con} \explain{} \\
$ \null \quad \textbf{con } m \cdot \\
\tabb v, f, q : \left[ 	
\begin{array}{l}
	pre(1) \land q = (N,s) \land |s| = 0 \land m = s,  \\
	pre(1) \land q = (N,s) \land s = xm \land |s| \neq 0 \land x = r_t
\end{array}
\right]$; \\
$ \tabb v, f, q : \left[ 	
\begin{array}{l}
	pre(1) \land q = (N,s) \land s = xm \land |s| \neq 0 \land x = r_t, \\
	post(1)
\end{array}
\right]$ \\ \\
%
\rc{enq} \explain{} \\
$ \null \quad enq(q, r_t)$; \\
$ \null \quad \nt{v, f, q : \left[ 	
\begin{array}{l}
	pre(1) \land q = (N, s) \land s = xm \land |s| \neq 0  \land x = r_t, \\
	post(1)
\end{array}
\right]}{(3)}$ \\ \\
%
Now we set our result flag $f$ to false so we can begin our traversal: \\ \\
%
(3) \rc{seq, ass} \explain{ note $\neg f$ is equivalent to $f = False$} \\
$ \null \quad f := False;$ \\
$ \null \quad \nt{v, f, q : \left[ 	
\begin{array}{l}
	pre(1) \land q = (N,s) \land s = xm \land |s| \neq 0 \land x = r_t \land \neg f, \\
	post(1)
\end{array}
\right]}{(4)}$ \\ \\
%
Now we conquer the more difficult task of refining our loop. We begin with a sequential composition: \\
%
(4) \rc{w-pre} \explain{Remove auxillary statements from our precondition with a w-pre} \\
$ \null \quad \nt{v, f, q : \left[ 	
\begin{array}{l}
	pre(1) \land q = (N,s) \land |s| \neq 0 \land \neg f, \\
	post(1)
\end{array}
\right]}{(5)}$ \\
%
And now we refine (5) into our main loop: \\
%
%
(5) \rc{while, isEmpty}  \\
\begin{algorithm}[H]
\While {$\neg (f \lor isEmpty(q))$} {
	$v, f, q : \nt{\left[
	\begin{array}{l}
		pre(1) \land q = (N,s) \land \neg f \land |s| \neq 0, \\
		pre(1) \land q = (N,s)
	\end{array}
	\right]}{(6)}$
} 
\end{algorithm}
%
\noindent \\
We derived our invariant based on the following properties of the queue:
\begin{itemize}
\item There exists a number $i$, such that the queue will be comprised of elements from the $i$'th and $(i + 1)$'th layer of the tree (in other words, $\Gamma_t^i(r_t)$ and $\Gamma_t^{i+1}(r_t)).$ 
\item For any element in $i$'th layer of the tree, if any of their successors are in the queue, then that element is not in the queue, and it is not the element we are searching for.
\item Or, we have found an element $w$ such that it's key matches the element we are searching for.
\end{itemize}
%
Where our loop invariant is:
%
\begin{equation*}
Inv : \left(
	\begin{array}{l}
		 \exists i \in \N .\Big(\forall p \in \sset{s} . (p \in \Gamma^i_t(r) \cup \Gamma^{i+1}_t(r))  \\ % In this level or the next
			  \land ~ \forall z \in \Gamma^{i}_t(r). p \in \Gamma(z) \implies (z \notin \sset{s} \land \kappa_t(z) \notin k) \Big) \\ % state that parents of any node in the queue not in the queue
			  \lor ~ ( \exists w \in V_t. ~w \in \Gamma_t^*(r_t) \land \kappa_t(w) = k \land \lambda_t(w) = v) % or, we found a thingo and we exit
		\end{array} \right)
\end{equation*} \\ 
There are a few cases of the state of the queue that were unnecessary to consider due to the definition of a tree.
We are able to represent the queue as a set directly as the elements in the queue at any given time will be unique, as no node can be added multiple times. This is due to the acyclic nature of the tree structure. \\ \\
(6) \rc{i-loc, seq x 2, con, c-frame} \explain{We split up our statement so we can perform deq() and whosNext(). Note here we've rewritten our queue sequence $s$ as $ys$, where $y = s_1 \land s = s_{2..}$. And by $s_1$ I mean the first element in the sequence s. This can be achieved as $|s| \neq 0$. We also remove q from the frame so that we can refine to whosNext()} \\ \\
$
\null \quad \textbf{con } y \cdot \textbf{var } e : V_t \cdot \\
\null \quad v, f, e : \left[ pre(1) \land q = (N, sy), pre(1) \land q = (N, sy) \land e = y \right] \\
\null \quad v, f, q, e : \left[ pre(1) \land q = (N, sy) \land e = y, pre(1) \land q = (N, s) \land e = y  \right] \\
\null \quad v, f, q, e : \left[ pre(6) \land e = y, post(6) \right]
$ \\ \\
Now we make the first step of getting a node to search over via \textsc{Deq}: \\ \\
(6) \rc{deq, whosNext} \explain{As we have n $\neq$ 0 in the precondition of (6) we can get the first element with whosNext() and remove it from the queue with deq()} \\ \\
$
\null \quad \textbf{var } e : V_t \cdot 
e := whosNext(q); \\
\tabb deq(q); \\ 
\null \quad \nt{v, f, q, e : \left[ pre(6) \land e = y, post(6) \right]}{(7)}
$ \\ \\
We now need a conditional to check if our current node $t$ is our goal. \\ \\
%
%
%
%
(7) \rc{if} \explain{} \\
\begin{algorithm}[H]
\eIf { $\kappa_t(e) = k$ } {
$\nt{v, f, q, e : \left[ pre(6) \land e = y \land \kappa_t(e) = k, post(6) \right]}{(8)}$
} {
$\nt{v, f, q, e : \left[ pre(6) \land e = y \land \kappa_t(e) \neq k, post(6) \right]}{(9)}$
}
\end{algorithm}
%
\noindent
(8) refines into our goal state, where we set the flag to true and assign the payload. \\ \\
%
%
(8) \rc{seq}  \explain{} \\
\tabb \quad $v, f, q, e : \left[ 
\begin{array}{l}
pre(6) \land e = y \land \kappa_t(e) = k, \\
pre(6) \land e = y \land \kappa_t(e) = k \land v = \lambda_t(t)
\end{array}
\right];$ \\ \\
\tabb \quad $v, f, q, e : \left[ ~pre(6) \land e = y \land \kappa_t(e) = k \land v = \lambda_t(e), post(6) \right]$ \\ \\
%
\tabb \rc{ass} \explain{}  \\
\tabb \quad  v := $\lambda_t(e)$; \\
\tabb \quad $v, f, q, e : \left[ pre(6) \land e = y \land \kappa_t(e) = k \land v = \lambda_t(e), post(6) \right]$ \\
%
\tabb \rc{s-post, ass} \explain{}  \\
\tabb \quad  v := $\lambda_t(e)$; \\
\tabb \quad  f := \textit{True} \\ \\
%
Now returning to (9), we need to enqueue all of the successors of the current node. Using the successor function $\Gamma$ to retrieve a set of all successors: \\ \\
%
(9) \rc{i-loc, seq, ass} \explain{Create a copy of the set of successors} \\
\tabb \textbf{var } succ $\cdot$
succ := $\Gamma_t(e)$; \\
\tabb $\nt{v, f, q, e, succ : \left[ pre(6) \land e = y \land \kappa_t(e) \neq k \land succ = \Gamma_t(e), post(6) \right]}{(10)}$ \\\\
%
Now looping through the set, we pick an element from the set each time and enqueue it in our queue.
% l is for leaf
\begin{equation*}
\begin{array}{l}
\end{array}
\end{equation*}
%
(10) \rc{while, seq x 2, ass x 2, enq} \\
\begin{algorithm}[H]
\While {succ $\neq \varnothing$} {
\textbf{var} $l :\in succ$ \\
enq(q, l) \\
$succ := succ \setminus \{ l\}$ 
}
\end{algorithm} 
%
Now, collecting our code we come to our BFS implementation using an abstract queue: \\

\begin{algorithm}[H]
\textbf{var} q : $\mathcal{Q};$ \\
\ab{init(q)}; \\
\ab{enq(q,$r_t$)}; \\
f := $False$; \\
\While {$\neg( f \lor \ab{isEmpty(q)})$} {
	\textbf{var} e : $V_t$; \\
	\ab{e := whosNext(q)}; \\
	\ab{deq(q)}; \\
	\eIf {$\kappa_t$(e) = k} {
		v := $\lambda_t$(e); \\
		f := $True$ \\
	} {
		\textbf{var} succ $\cdot$ succ := $\Gamma_t$(e); \\
		\While {succ $\neq \varnothing$ } {
			\textbf{var} l :$\in$ succ; \\
			\ab{enq(q, l)}; \\
			succ := succ $\setminus \{l\}$ \\
		}
	}
}
\end{algorithm}
%
Note here our \ab{abstract} queue operations are highlighted in \ab{Wild Strawberry}.
%
%
%%%%%%%%%%%%%%%%%%%%%%%%%%%%%%%%%%%%%%%%%%%%%%%%%%%%%%%%%%%%%%%%%
%
%
\section{Refining our abstract implementation of a queue}
%
\subsection{Step 1}
%
We begin by introducing new variables to describe our concrete implementation. We require: \\
%
\begin{itemize}
\item An array $A : V_t^{N+1}$ to store our queue elements.
\item An enqueue counter $n : \N$ to mark where we push elements to the queue.
\item A dequeue counter $m : \N$ to mark where we pop elements from the queue.
\end{itemize}
%
Note that both $n$ and $m$ are bounded by the conditions $0 \leq n < N + 1$ and $0 \leq m < N + 1$ respectively.
%
\subsection{Step 2}
%
Now we define our coupling invariant to relate our concrete and abstract implementations:
%
 \[ \C : \Q \times V_t^* \times \N \times \N \to \mathbb{B} \]
\begin{align*}
 \C((N, s), A, n, m) = &\Big( \forall i \in \Big[0..(n - m)(mod ~N + 1)\Big). ~A[(n + i)(mod ~N + 1)] = s_i\Big) \\
 &\land |s| = (n-m)(mod N + 1) 
\end{align*}
%
%
\subsection{Step 3}
%
Now we augment our new assignments in order to re-establish the coupling invariant $\C$: \\
%
%
\begin{algorithm}[H]
\textbf{var} q : $\mathcal{Q};$ \\
\ab{init(q)}; \textcolor{green}{\tabb (A)}\\ 
\con{
\textbf{var} A : $V_t^{N+1}$; 
\textbf{var} n : $\N$;
\textbf{var} m : $\N$;
} \\
\con {
n := 0; m := 0;
} \\
\ab{enq(q,$r_t$)}; \textcolor{green}{\tabb (B)}\\
\con{A[n] := $r_t$; \\
n := n + 1(mod N + 1);} \\
f := $False$; \\
\While {$\neg( f \lor \ab{isEmpty(q)})$} {
	\textbf{var} e : $V_t$; \\
	\ab{e := whosNext(q)}; \\
	\con{e := A[m];} \\
	\ab{deq(q)}; \textcolor{green}{\tabb (C)}\\
	\con{m := m + 1 (mod N + 1);} \\
	\eIf {$\kappa_t$(e) = k} {
		v := $\lambda_t$(e); \\
		f := $True$ \\
	} {
		\textbf{var} succ := $\Gamma_t$(e); \\
		\While {succ $\neq \varnothing$ } {
			\textbf{var} l :$\in$ succ; \\
			\ab{enq(q, l)}; \textcolor{green}{\tabb (D)}\\
			\con{
			A[n] := l; n := n + 1(mod N + 1);
			} \\
			succ := succ $\setminus \{l\}$ \\
		}
	}
}
\end{algorithm}
%
%
We now must prove that our coupling invariant holds after each operation and after our initialisation. We are obliged to prove \textcolor{green}{(A-D)}.\\ \\
%
\textcolor{green}{(A)}
We must refine: \\
%
$q, A, m, n : [True, q = (N,s) \land |s| = 0 \land \C(q,A,m,n)]$ \\ \\ 
%
\rc{seq, c-frame} \\
$q : [True, q = (N,s) \land |s| = 0];$ \\
$A, m, n : [ q = (N,s) \land |s| = 0, \C((N,s),A,m,n)]$ \\ \\
\rc{initq} \\
initq(q); \\
$A, m, n : [ q = (N,s) \land |s| = 0, \C((N,s),A,m,n)]$ \\  \\
\rc{s-post, ass x 2} \explain{Our coupling invariant allows for any arbtrary $m$ such that $m$ = $n$ to represent an empty queue. By using an s-post to add $m = 0$ into the post condition, we can initialise $m$ and follow up by setting $n$ to the same value.} \\
initq(q); \\
m := 0; n := 0; \\ \\
%
%
\textcolor{green}{(B)}
We must refine the following for our initial enqueue: \\ 
$q, A, m, n : [\C(q,A,m,n), \C(q,A,m,n) \land q = (N,s) \land s = xs_0 \land |s| \neq 0]$ \\ \\
%
%
\rc{i-con, c-frame, seq} \\
\textbf{con } S $\cdot$ $q : \left[
\begin{array}{l}
\C((N,S),A,m,n) \land s = S, \\
s = xS \land |s| \neq 0 \land q = (N,s) \land \C((N,S),A,m,n)
\end{array}
\right];$ \\
%
%
\tabb $~A, m, n : \left[
\begin{array}{l}
\C((N,S),A,m,n) \land  s = xS \land |s| \neq 0 \land q = (N,s), \\
\C(q,A,m,n) 
\end{array}
\right]$ \\
%
%
\rc{enq} \\
\textbf{con } S $\cdot$ enq(q,x);\\
\tabb $~A, m, n : \left[
\begin{array}{l}
\C((N,S),A,m,n) \land  s = xS \land |s| \neq 0 \land q = (N,s), \\
\C(q,A,m,n) 
\end{array}
\right]$ \\ \\
%
%
\rc{a-ass, ass} \explain{To re-establish our coupling invariant, we must update out array at position $n$ to match the corresponding element in the sequence (as described in the first conjunct) and then increment $n$ to $n + 1 (mod~N+1)$ to ensure that $|s| = (n - m)(mod N + 1)$ (as described in the second conjunct).} \\
\textbf{con } S $\cdot$ enq(q,x);\\
A[n] := x; n := n + 1 (mod N+1) \\ \\
%
%
\textcolor{green}{(C)}
We must now refine our deq() in the loop: \\ \\
%
$ q, A, m, n : [\C((N,sy),A,m,n) \land |sy| \geq 0 \land q = (N,sy), \C((N,s),A,m,n)) \land s= s_0 ];$ \\  \\
%
\rc{seq, c-frame, i-con} \explain{Removing everything but our dequeue counter from the frame}\\
\textbf{con} S $\cdot$
$ q : [\C((N,sy),A,m,n) \land |sy| \geq 0 \land q = (N, sy) \land S = s, (\C((N,Sy),A,m,n)) \land q = (N,S) ];$ \\
$ m : [\C((N,Sy),A,m,n) \land |sy| \geq 0, \C((N,S),A,m,n)) ];$ \\ \\ \\
%
\rc{deq} \\
deq(q); \\
$ m : [\C((N,Sy),A,m,n) \land |sy| \geq 0, \C((N,S),A,m,n)) ];$ \\ \\
%
\rc{ass} \explain{To re-establish our coupling invariant in the post condition, we must match the new sequence established in the abstract stack. To accomplish this, we increment $m$ to $m (mod N+1)$ such that $n-m = |s|$} \\
deq(q); \\
m := m + 1 (mod N + 1); \\ \\
%
\textcolor{green}{(D)} Finally, the proof of the final enque follows that of \textcolor{green}{(B)}.
%
\subsection{Step 4}
We now replace boolean expressions involving our abstract queue with concrete ones. We only have 1 expression involving our abstract queue, \ab{isEmpty()}, which we can show is equivalent to \con{$m = n$}. To show this, we bring in our coupling invariant with $|s|$ = 0:
\[ \C((N, s), A, n, m) = \Big( \forall i \in [0..0). ~A[(m + i)(mod ~N + 1)] = s_i\Big) \land 0 = 0 \]
%
Clearly, $[0,0)$ gives us no $i$ to describe the array, i.e. we have no elements on the LHS. As our sequence is empty, there are no $n \in \N$ such that $s_n$ is an element, therefore we have an empty sequence, so our coupling invariant is satisfied.\\

\noindent
Now we may replace \ab{isEmpty()} with \con{$m = n$}. \\ \\

\noindent
Next, we must show that whosNext() follows our coupling invariant: 

\subsection{Step 5}

Our program is now independant of our abstract queue, and we can remove auxillary queue references. We now arrive at our concrete implementation:


\begin{algorithm}[H]
\con{
\textbf{var} A : $V_t^{N+1}$; 
\textbf{var} n : $\N$;
\textbf{var} m : $\N$;
} \\
\con {
n := 0; m := 0;
} \\
\con{A[n] := $r_t$; \\
n := n + 1(mod N + 1);} \\
f := $False$; \\
\While {$\neg( f \lor \con{m = n})$} {
	\textbf{var} e : $V_t$; \\
	\con{e := A[m];} \\
	\con{m := m + 1 (mod N + 1);} \\
	\eIf {$\kappa_t$(e) = k} {
		v := $\lambda_t$(e); \\
		f := $True$ \\
	} {
		\textbf{var} succ := $\Gamma_t$(e); \\
		\While {succ $\neq \varnothing$ } {
			\textbf{var} l :$\in$ succ; \\
			\con{
			A[n] := l; n := n + 1(mod N + 1);
			} \\
			succ := succ $\setminus \{l\}$ \\
		}
	}
}
\end{algorithm}

Yea the implementation! \\

\section{C code}

\lstinputlisting{bfs.c}

\end{document}
