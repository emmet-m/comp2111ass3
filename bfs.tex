\documentclass[a4paper]{scrartcl}
\usepackage[l2tabu,orthodox]{nag}% Old habits die hard. All the same, there are commands, classes and packages which are outdated and superseded. nag provides routines to warn the user about the use of those.

\usepackage[all,error]{onlyamsmath}% Error on deprecated math commands like $$ $$.
\usepackage{fixltx2e}
\usepackage[strict=true]{csquotes}

\usepackage{color}
\usepackage[colorlinks=true]{hyperref}
\usepackage{2111defs,2111theorems}
\title{\texttt{$\mathbb{C}$ase $\mathbb{S}$tudy: $\mathbb{H}$elp us oh $\mathcal{HELP}$}}
\author{$\mathbb{E}$mmet $\mathbb{M}$urray z5059840, $\mathbb{D}$anni $\mathbb{O}$vens z5059491}

% Danni's awesome and super helpful math things y@y
\newcommand{\N}{\mathbb{N}}
\newcommand{\Z}{\mathbb{Z}}
\newcommand{\refinedby}{\sqsubseteq} % THIS IS MY PERSONAL FAV ~refined by~
\newcommand\textlcsc[1]{\textsc{\MakeLowercase{#1}}} % pretty small uppercase letters

% Document starts here
\begin{document}
\maketitle
\section*{Task 1}
We will begin by defining some abstract queue operations in our toy language:\\
%
\begin{center}
{\LARGE{\textsc{Queue:}}\normalsize}
\end{center}
\begin{align*}
\mathcal{Q} &:: \quad \langle ~ \rangle \\
&~| \quad \langle v : val,~\mathit{next} : \mathcal{Q} \rangle \\
\end{align*}
%
\begin{align*}
% Feel free to change the formatting I was just playing around
% INITQ
& \textbf{proc }\text{\textsc{initq}(q, val N) } \cdot \\ 
    &\quad \refinedby  q : [True, q = \langle \rangle ] \\
    & \quad \refinedby q := \langle \rangle \\
\end{align*}
\begin{align*}
& \textbf{func }\text{\textsc{isEmpty}(q) } : \text{boolean} \cdot  \\ 
	& \quad \textbf{var } b; \\
	& \quad ~ q : [True,~b \iff q = \langle \rangle  ] \\
	& \quad \text{return b} \\
		%
	& \quad \refinedby \textbf{var } b; \\
	& \quad \textbf{if } q = \langle \rangle \textbf{ then} \\
	& \quad \quad b := True \\
	& \quad \textbf{else} \\
	& \quad \quad b := False \\
	& \quad \textbf{fi} \\
	& \quad \text{return b} \\
\end{align*}
\begin{align*}
% ENQ
& \textbf{proc }\text{\textsc{enq}(q, val v) } \cdot\\
	& \quad q : \left[|q| < N, q = \langle v,~q_0 \rangle \right] \\
    & \quad \refinedby \text{ q := $\langle$ v, q $\rangle$}\\
\end{align*}
\begin{align*}
% Dis what I mean by dequeue
& \textbf{func }\text{\textsc{deq}(q : $\mathcal{Q}$) :} \text{ val } \cdot \\
	% post: there exists a Queue r in q_0 such that r's next is <> implies that t will be r's value and q is the same as q_0, but r is <>
    & \quad \quad \textbf{var }t : \text{ val} \\
	& \quad \quad ~q : \left[ |q| > 0, \exists~r \in q_0. r.next = \langle \rangle \implies t = r.value \land q = q_0[^r/_{\langle \rangle}] \right] \\
    & \quad \quad \textbf{ return } t \\
	%%%%%%%%%%%
    & \quad \refinedby \textbf{var }t : \text{ val} \\
    & \quad \quad \textbf{ if } q.next = \langle \rangle \textbf{ then} \\
	& \quad \quad \quad \text{t := q.v;} \\
    & \quad \quad \textbf{ else} \\
	& \quad \quad \quad \text{t := deq(q.next);} \\
    & \quad \quad \textbf{ fi} \\
    & \quad \quad \textbf{ return } t \\\\
\end{align*}

\section*{Task 2}

\end{document}
