\documentclass[a4paper]{scrartcl}
\usepackage[l2tabu,orthodox]{nag}% Old habits die hard. All the same, there are commands, classes and packages which are outdated and superseded. nag provides routines to warn the user about the use of those.

\usepackage[all,error]{onlyamsmath}% Error on deprecated math commands like $$ $$.
\usepackage{fixltx2e}
\usepackage[strict=true]{csquotes}
\usepackage[usenames, dvipsnames]{color}
\usepackage[colorlinks=true]{hyperref}
\usepackage{2111defs2,2111theorems}

% Algortihms package for pretty code formatting!
\usepackage{algorithm2e}
\usepackage{algpseudocode}

\allowdisplaybreaks

\title{\texttt{$\mathbb{C}$ase $\mathbb{S}$tudy: $\mathbb{H}$elp us oh $\mathcal{HELP}$}}
\author{$\mathbb{E}$mmet $\mathbb{M}$urray z5059840, $\mathbb{D}$anni $\mathbb{O}$vens z5059491}

% Danni's awesome and super helpful math things y@y
\newcommand{\N}{\mathbb{N}}
\newcommand{\Z}{\mathbb{Z}}
\newcommand{\refinedby}{\sqsubseteq} % THIS IS MY PERSONAL FAV ~refined by~
\newcommand\textlcsc[1]{\textsc{\MakeLowercase{#1}}} % pretty small uppercase letters
%This is actually the best thing we've probably ever done
\newcommand{\rc}[1]{ $\refinedby$ \quad \textbf{\textcolor{ForestGreen}{$\langle$ #1 $\rangle$}}}

%I cna't live without this:
\newcommand{\squigglyBoy}{~}
\newcommand{\leftcurlyBoy}{\{}
\newcommand{\rightcurlyBoy}{\}}

% This is just fun:
\newcommand{\que}[1]{\langle #1 \rangle}
\newcommand{\explain}[1]{\textcolor{RoyalBlue}{\textit{#1}}}

% Document starts here
\begin{document}
\maketitle
\section{BFS Implementation}
%
\subsection{Defining an abstract queue}
We will begin by defining some abstract queue operations in our toy language: \\
\begin{center}
{\LARGE{\textsc{Queue:}}\normalsize}
\end{center}
\begin{align*}
\mathcal{Q} &:: \quad \langle ~ \rangle \\
&~| \quad \langle v : val,~\mathit{next} : \mathcal{Q} \rangle \\
\end{align*}
Note: q.n refers to the length of the queue, as defined in the specification task 1.
%
% Feel free to change the formatting I was just playing around
% INITQ
 $\textbf{proc }\text{\textsc{initq}(q, val N) } \cdot \\ 
	\quad \quad  q : [True, q = \que{} \land q.n = 0  ] \\
	\quad \quad \refinedby q := \langle \rangle \\\\
 $
% ISEMPTY
 $\textbf{func }\text{\textsc{isEmpty}(q) } : \text{boolean} \cdot  \\ 
	 \quad \textbf{var } b; \\
	 \quad ~ q : [True,~b \iff q = \langle \rangle  ] \\
	 \quad \text{ return b} \\
 $
%
%
$\refinedby$  \textbf{   var } b; \\
\begin{algorithm}[H]
\eIf {$q = \langle \rangle$} {
$
b := True 
$
}{
$
b := False
$
} 
\end{algorithm} 
% ENQ
$
 \textbf{proc }\text{\textsc{enq}(q, val v) } \cdot\\
	 \quad q : \left[q.n < N, q = \langle v,~q_0 \rangle \right] \\
     \quad \refinedby \text{ q := $\langle$ v, q $\rangle$}\\\\
% Dis what I mean by dequeue
 \textbf{func }\text{\textsc{deq}(q : $\mathcal{Q}$) :} \text{ val } \cdot \\
	% post: there exists a Queue r in q_0 such that r's next is <> implies that t will be r's value and q is the same as q_0, but r is <>
     \quad \quad \textbf{var }t : \text{ val} \\
	 \quad \quad ~q, t : \left[ q.n > 0, \exists r \in q_0.( r.next = \langle \rangle \implies t = r.value \land q = q_0[^{\langle \rangle}/_r]) \right] \\
     \quad \quad \textbf{ return } t \\\\
	%%%%%%%%%%%
     \quad \refinedby \textbf{var }t : \text{ val} \\
     \quad \quad \textbf{ if } q.next = \langle \rangle \textbf{ then} \\
		 \quad \quad \quad \text{t := q.v;} \\
     \quad \quad \textbf{ else} \\
		 \quad \quad \quad \text{t := deq(q.next);} \\
     \quad \quad \textbf{ fi} \\
     \quad \quad \textbf{ return } t \\\\
$
%
\subsection{Refinement}
We begin by refining the provided specification of \textsc{Search}: \\
%
\textbf{proc }{\textsc{Search}(\textbf{value} $t$, \textbf{value} $N$, \textbf{value} $k$, \textbf{result} $v$, \textbf{result} $f$) $\cdot$\\
\[ t, N, k, v, f: \left[
    \begin{array}{l} 
	\textsc{Tree}(t) \land \text{max}_{i\in\N} | \Gamma^i_t (r_t)  \cup \Gamma^{i+1}_t (r_t) | \leq N ,\\
	(f \land \exists w \in V_{t_0} (\kappa_{t_0}(w) = k_0 \land \lambda_{t_0} (w) = v)) ~ \lor \\
	(\neg f  \land \forall w \in V_{t_0} (\kappa_{t_0}(w) \neq k_0))
    \end{array} 
\right] \]
%
%
\rc{c-frame} \\ \\
$\nt{v, f: \left[
    \begin{array}{l} 
	\textsc{Tree}(t) \land \text{max}_{i\in\N} | \Gamma^i_t (r_t)  \cup \Gamma^{i+1}_t (r_t) | \leq N ,\\
	(f \land \exists w \in V_{t} (\kappa_{t}(w) = k \land \lambda_{t} (w) = v)) ~ \lor \\
	(\neg f  \land \forall w \in V_{t} (\kappa_{t}(w) \neq k))
    \end{array} 
\right]}{(1)}$ \\ \\
%
We neeeeeeed a queue to do operations with, so we create a queue variable and initialise it: \\
\rc{i-loc} \explain{q doesn't occur yet} \\
\textbf{var } q : $\mathcal{Q}~\cdot$ 
$v, f, q \left[ 
~pre(1),~
post(1)~
\right]$ \\ \\
%
%
%
\rc{seq} \explain{so we can refine to our procedure call}\\
$v, f, q : \left[ 
	\squigglyBoy True, \squigglyBoy
	pre(1) \land q = \que{} \land q.n = 0
\right]$; \\
$v, f, q : \left[ 	
	pre(1) \land q = \que{} \land q.n = 0, \squigglyBoy
	post(1)
\right]$ \\ \\
%
%
%
\rc{proc} \\
\textsc{initq}(q,N); \\
$\nt{v, f, q : \left[ 	
	pre(1) \land q = \que{} \land q.n = 0, \squigglyBoy
	post(1)
\right]}{(2)}$ \\

\noindent
Now that we have our queue initialised we must add our root node, $t$. to begin searching: \\
(2) \rc{seq, i-con} \explain{We introduce a constant c that freezes our queue's initial state} \\
\textbf{con } c $\cdot$ 
$v, f, q : \left[ 	
\begin{array}{l}
	pre(1) \land q = \que{} \land q.n = 0 \land q = c, \squigglyBoy \\
	q = \que{t, c} \land pre(1)
\end{array}
\right]$; \\
% Do't break up the \squigglyBoy fam please.
$~~~~~~~~~~~~~v, f, q : \left[ 	
	q = \que{t, c} \land pre(1), \squigglyBoy
	post(1)
\right]$ \\

\noindent
\rc{proc, s-post} \explain{As our pre and post conditions are subsets of \textsc{enq}'s, we refine to the procedure. As we previously set $c = \que{}$, we can replace c with the empty queue. We can now get rid of c.} \\
%
%
\textsc{enq}(q, t); \\
% Do't break up the \squigglyBoy fam please.
$ \nt{v, f, q : \left[ 	
	q = \que{t, \que{}} \land pre(1), \squigglyBoy
	post(1)
\right]}{(3)}$

\end{document}
























 ALGORITHM README

\begin{algorithm}[H]
 %Condition inside first brackets
\eIf {$$} {
 If part follows
$
$
}{
 Else part
$
$
}
\end{algorithm}


