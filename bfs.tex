\documentclass[a4paper]{scrartcl}
\usepackage[l2tabu,orthodox]{nag}% Old habits die hard. All the same, there are commands, classes and packages which are outdated and superseded. nag provides routines to warn the user about the use of those.

\usepackage[all,error]{onlyamsmath}% Error on deprecated math commands like $$ $$.
\usepackage{fixltx2e}
\usepackage[strict=true]{csquotes}
\usepackage[usenames, dvipsnames]{color}
\usepackage[colorlinks=true]{hyperref}
\usepackage{2111defs2,2111theorems}

% Algortihms package for pretty code formatting!
\usepackage{algorithm2e}
\usepackage{algpseudocode}

\allowdisplaybreaks 

\title{\texttt{$\mathbb{C}$ase $\mathbb{S}$tudy: $\mathbb{H}$elp us oh $\mathcal{HELP}$}}
\author{$\mathbb{E}$mmet $\mathbb{M}$urray z5059840, $\mathbb{D}$anni $\mathbb{O}$vens z5059491}

% Danni's awesome and super helpful math things y@y
\newcommand{\N}{\mathbb{N}}
\newcommand{\Z}{\mathbb{Z}}
\newcommand{\refinedby}{\sqsubseteq} % THIS IS MY PERSONAL FAV ~refined by~
\newcommand\textlcsc[1]{\textsc{\MakeLowercase{#1}}} % pretty small uppercase letters
%This is actually the best thing we've probably ever done
\newcommand{\rc}[1]{ $\refinedby$ \quad \textbf{\textcolor{ForestGreen}{$\langle$ #1 $\rangle$}}}
\newcommand{\sset}[1]{ \llcorner #1 \lrcorner }

%I cna't live without this:
\newcommand{\squigglyBoy}{~}
\newcommand{\leftcurlyBoy}{\{}
\newcommand{\rightcurlyBoy}{\}}


% This is just fun:
\newcommand{\que}[1]{\langle #1 \rangle}
\newcommand{\explain}[1]{\textcolor{RoyalBlue}{\textit{#1}}}
\newcommand{\tabb}{\null \quad}
\newcommand{\tabbb}{\null \quad \quad}
\newcommand{\tabbbb}{\null \quad \quad \quad}

% Document starts here
\begin{document}
\maketitle
\section{BFS Implementation}
%
\subsection{Defining an abstract queue}
We will begin by defining some abstract queue operations in our toy language: \\
\begin{center}
{\LARGE{\textsc{Queue:}}\normalsize}
\end{center}
\begin{equation*}
\mathcal{Q} :: ( N:\N, ~s:V_{t}^{*} ) \\
\end{equation*} \\
%
Where $N$ is the max size of the queue, $n$ is the current size, and $s$ is a sequence of queue values. \\
%
We introduce the notation of $\llcorner s \lrcorner$, meaning to treat the sequence $s$ as a set such that all the values of s are in the $\llcorner s \lrcorner$. For convenience, we define $|s|$ as equivalent to $|\llcorner s \lrcorner|$.\\
%
We can now define our 5 core abstract queue operations: \\ \\
%
\textbf{Initq:} \\
% Do we even need this? In Liam's example he just sets the queue to the empty queue.
$q : [True, ~ q = ( N, s) \land |s| = 0] \refinedby initq(q)$ \\
\textbf{Enq:} \\
$q : [n <= N, ~ q = ( N, s) \land s = xs_0 \land |s| \neq 0] \refinedby enq(q, x)$ \\
\textbf{Deq:} \\
$q : [n > 0 \land q = (N, xs), q = (N, s) \land s = s_0 ] \refinedby  deq(q)$ \\
\textbf{WhosNext:} \\
$q,x : [q = (N, sy), q = q_0 \land x = y] \refinedby x := whosNext(q)$ \\
\textbf{isEmpty:} \\
$b : [q = (N, s), b \iff |s| = 0 ] \refinedby b := isEmpty(q)$\\
%
\subsection{Refinement}
We begin by refining the provided specification of \textsc{Search}: \\ \\
%
\textbf{proc }{\textsc{Search}(\textbf{value} $t$, \textbf{value} $N$, \textbf{value} $k$, \textbf{result} $v$, \textbf{result} $f$) $\cdot$\\
\[ t, N, k, v, f: \left[
    \begin{array}{l} 
	\textsc{Tree}(t) \land \text{max}_{i\in\N} | \Gamma^i_t (r_t)  \cup \Gamma^{i+1}_t (r_t) | \leq N ,\\
	(f \land \exists w \in V_{t_0} (\kappa_{t_0}(w) = k_0 \land \lambda_{t_0} (w) = v)) ~ \lor \\
	(\neg f  \land \forall w \in V_{t_0} (\kappa_{t_0}(w) \neq k_0))
    \end{array} 
\right] \] \\
%
%
\rc{c-frame} \\ 
$\null \quad \nt{v, f: \left[
    \begin{array}{l} 
	\textsc{Tree}(t) \land \text{max}_{i\in\N} | \Gamma^i_t (r_t)  \cup \Gamma^{i+1}_t (r_t) | \leq N ,\\
	(f \land \exists w \in V_{t} (\kappa_{t}(w) = k \land \lambda_{t} (w) = v)) ~ \lor \\
	(\neg f  \land \forall w \in V_{t} (\kappa_{t}(w) \neq k))
    \end{array} 
\right]}{(1)}$ \\ \\
%
\\ In order to use a breadth first search along the tree, we need a queue. Thus we must create a queue variable and initialise it to an empty queue: \\  \\
%I LOC
(1) \rc{i-loc} \explain{q doesn't occur yet} \\
\null \quad \textbf{var } q : $\mathcal{Q}~\cdot$ 
$v, f, q \left[ 
~pre(1),~
post(1)~
\right]$ \\ \\
%
%
% SEQ
\rc{seq} \explain{so we can refine to initq}\\
$ \null \quad v, f, q : \left[ 
	~ pre(1), ~
	pre(1) \land q = (N,s) \land |s| = 0
\right]$; \\
$ \null \quad v, f, q : \left[ 	
	pre(1) \land q = (N,s) \land |s| = 0, ~
	post(1)
\right]$ \\ \\
%
%
\rc{initq} \explain{}\\
$ \null \quad initq(q);$ \\
$ \null \quad \nt{v, f, q : \left[ 	
	pre(1) \land q = (N,s) \land |s| = 0, ~
	post(1)
\right]}{(2)}$ \\ \\
%
%
Similarly, we push our initial element $r_t$ onto the queue: \\
%
(2) \rc{seq, con} \explain{} \\
$ \null \quad \textbf{con } m \cdot \\
\tabb v, f, q : \left[ 	
\begin{array}{l}
	pre(1) \land q = (N,s) \land |s| = 0 \land m = s,  \\
	pre(1) \land q = (N,s) \land s = xm \land |s| \neq 0 \land x = r_t
\end{array}
\right]$; \\
$ \tabb v, f, q : \left[ 	
\begin{array}{l}
	pre(1) \land q = (N,s) \land s = xm \land |s| \neq 0 \land x = r_t, \\
	post(1)
\end{array}
\right]$ \\ \\
%
\rc{enq} \explain{} \\
$ \null \quad enq(q, r_t)$; \\
$ \null \quad \nt{v, f, q : \left[ 	
\begin{array}{l}
	pre(1) \land q = (N, s) \land s = xm \land |s| \neq 0  \land x = r_t, \\
	post(1)
\end{array}
\right]}{(3)}$ \\ \\
%
Now we set our result flag $f$ to false so we can begin our traversal: \\ \\
%
(3) \rc{seq, ass} \explain{ note $\neg f$ is equivalent to $f = False$} \\
$ \null \quad f := False;$ \\
$ \null \quad \nt{v, f, q : \left[ 	
\begin{array}{l}
	pre(1) \land q = (N,s) \land s = xm \land |s| \neq 0 \land x = r_t \land \neg f, \\
	post(1)
\end{array}
\right]}{(4)}$ \\ \\
%
Now we conquer the more difficult task of refining our loop. We begin with a sequential composition: \\
%
(4) \rc{w-pre} \explain{Remove auxillary statements from our precondition with a w-pre} \\
$ \null \quad \nt{v, f, q : \left[ 	
\begin{array}{l}
	pre(1) \land q = (N,s) \land |s| \neq 0 \land \neg f, \\
	post(1)
\end{array}
\right]}{(5)}$ \\
%
And now we refine (5) into our main loop: \\
%
%
(5) \rc{while, isEmpty}  \\
\begin{algorithm}[H]
\While {$\neg (f \lor isEmpty(q))$} {
	$v, f, q : \nt{\left[
	\begin{array}{l}
		pre(1) \land q = (N,s) \land \neg f \land |s| \neq 0, \\
		pre(1) \land q = (N,s)
	\end{array}
	\right]}{(6)}$
} 
\end{algorithm}

Where our loop invariant is:

\begin{equation*}
Inv : \left(
	\begin{array}{l}
		 \exists i \in \N .\Big(\forall p \in \sset{s} . (p \in \Gamma^i_t(r) \cup \Gamma^{i+1}_t(r))  \\
			  \land ~ \forall z \in \Gamma^{i}_t(r). p \in \Gamma(z) \implies (z \notin \sset{s} \land \kappa_t(z) \notin k) \Big) \\
			  \lor ~ ( \exists w \in V_t. ~w \in \Gamma_t^*(r_t) \land \kappa_t(w) = k \land \lambda_t(w) = v)
		\end{array} \right)
\end{equation*} \\ 
There are a few cases of the state of the queue that were unnecessary to consider due to the definition of a tree.
We are able to represent the queue as a set directly as the elements in the queue at any given time will be unique, as no node can be added multiple times. This is due to the acyclic nature of the tree structure. \\ \\
(6) \rc{i-loc, seq x 2, con} \explain{We split up our statement so we can perform deq() and whosNext(). Note here we've rewritten our queue sequence $s$ as $ys$, where $y = s_1 \land s = s_{2..}$. And by $s_1$ I mean the first element in the sequence s. This can be achieved as $|s| \neq 0$.} \\ \\
$
\null \quad \textbf{con } y \cdot \textbf{var } e \cdot \\
\null \quad v, f, q, e : \left[ pre(1) \land q = (N, sy), pre(1) \land q = (N, sy) \land e = y \right] \\
\null \quad v, f, q, e : \left[ pre(1) \land q = (N, sy) \land e = y, pre(1) \land q = (N, s) \land e = y  \right] \\
\null \quad v, f, q, e : \left[ pre(6) \land e = y, post(6) \right]
$ \\ \\
Now we make the first step of getting a node to search over via \textsc{Deq}: \\ \\
(6) \rc{deq, whosNext} \explain{As we have n $\neq$ 0 in the precondition of (6) we can get the first element with whosNext() and remove it from the queue with deq()} \\ \\
$
\null \quad \textbf{var } t \cdot 
t := whosNext(q); \\
\tabb deq(q); \\ 
\null \quad \nt{v, f, q, t : \left[ pre(6) \land e = y, post(6) \right]}{(7)}
$ \\ \\
We now need a conditional to check if our current node $t$ is our goal. \\ \\
%
%
%
%
(7) \rc{if} \explain{} \\
\begin{algorithm}[H]
\eIf { $\kappa_t(e) = k$ } {
$\nt{v, f, q, e : \left[ pre(6) \land e = y \land \kappa_t(e) = k, post(6) \right]}{(8)}$
} {
$\nt{v, f, q, e : \left[ pre(6) \land e = y \land \kappa_t(e) \neq k, post(6) \right]}{(9)}$
}
\end{algorithm}
%
\noindent
(8) refines into our goal state, where we set the flag to true and assign the payload. \\ \\
%
%
(8) \rc{seq}  \explain{} \\
\tabb \quad $v, f, q, e : \left[ 
\begin{array}{l}
pre(6) \land e = y \land \kappa_t(e) = k, \\
pre(6) \land e = y \land \kappa_t(e) = k \land v = \lambda_t(t)
\end{array}
\right];$ \\ \\
\tabb \quad $v, f, q, e : \left[ ~pre(6) \land e = y \land \kappa_t(e) = k \land v = \lambda_t(e), post(6) \right]$ \\ \\
%
\tabb \rc{ass} \explain{}  \\
\tabb \quad  v := $\lambda_t(e)$; \\
\tabb \quad $v, f, q, e : \left[ pre(6) \land e = y \land \kappa_t(e) = k \land v = \lambda_t(e), post(6) \right]$ \\
%
\tabb \rc{s-post, ass} \explain{}  \\
\tabb \quad  v := $\lambda_t(e)$; \\
\tabb \quad  f := \textit{True} \\ \\
%
Now returning to (9), we need to enqueue all of the successors of the current node. Using the successor function $\Gamma$ to retrieve a set of all successors: \\ \\
%
(9) \rc{i-loc, seq, ass} \explain{Create a copy of the set of successors} \\
\tabb \textbf{var } succ $\cdot$
succ := $\Gamma_t(e)$; \\
\tabb $\nt{v, f, q, e, succ : \left[ pre(6) \land e = y \land \kappa_t(e) \neq k \land succ = \Gamma_t(e), post(6) \right]}{(10)}$ \\\\
%
Now looping through the set, we pick an element from the set each time and enqueue it in our queue.
% l is for leaf
\begin{equation*}
\begin{array}{l}
\end{array}
\end{equation*}
%
(10) \rc{while, seq x 2, ass x 2, enq} \\
\begin{algorithm}[H]
\While {succ $\neq \varnothing$} {
\textbf{var} $l :\in succ$ \\
enq(q, l) \\
$succ := succ \setminus \{ l\}$ 
}
\end{algorithm} 
%
Now, collecting our code we come to our BFS implementation using an abstract queue: \\

\begin{algorithm}[H]
\textbf{var} q : $\mathcal{Q} \cdot$ \\
init(q); \\
enq(q,$r_t$); \\
f := $False$; \\
\While {$\neg( f \lor isEmpty(q))$} {
	\textbf{var} e$\cdot$ 
	e := whosNext(q); \\
	deq(q); \\
	\eIf {$\kappa_t$(e) = k} {
		v := $\lambda_t$(e); \\
		f := $True$ \\
	} {
		\textbf{var} succ $\cdot$ succ := $\Gamma_t$(e); \\
		\While {succ $\neq \varnothing$ } {
			\textbf{var} l :$\in$ succ; \\
			enq(q, l); \\
			succ := succ $\setminus \{l\}$ \\
		}
	}
}
\end{algorithm}


\section{Refining our abstract implementation of a queue}
\end{document}



 ALGORITHM README

\begin{algorithm}[H]
 %Condition inside first brackets
\eIf {$$} {
 If part follows
$
$
}{
 Else part
$
$
}
\end{algorithm}
